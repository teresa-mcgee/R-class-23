%%%%%%%%%%%%%%%%%%%%%%%%%%%%%%%%%%%%%%%%%
% Short Sectioned Assignment
% LaTeX Template
% Version 1.0 (5/5/12)
%
% This template has been downloaded from:
% http://www.LaTeXTemplates.com
%
% Original author:
% Frits Wenneker (http://www.howtotex.com)
%
% License:
% CC BY-NC-SA 3.0 (http://creativecommons.org/licenses/by-nc-sa/3.0/)
%
%%%%%%%%%%%%%%%%%%%%%%%%%%%%%%%%%%%%%%%%%

%----------------------------------------------------------------------------------------
%	PACKAGES AND OTHER DOCUMENT CONFIGURATIONS
%----------------------------------------------------------------------------------------

\documentclass[paper=a4, fontsize=11pt]{scrartcl} % A4 paper and 11pt font size

\usepackage[T1]{fontenc} % Use 8-bit encoding that has 256 glyphs
\usepackage{fourier} % Use the Adobe Utopia font for the document - comment this line to return to the LaTeX default
\usepackage[english]{babel} % English language/hyphenation
\usepackage{amsmath,amsfonts,amsthm} % Math packages

\usepackage{lipsum} % Used for inserting dummy 'Lorem ipsum' text into the template

\usepackage{sectsty} % Allows customizing section commands
\allsectionsfont{\centering \normalfont\scshape} % Make all sections centered, the default font and small caps

\usepackage{fancyhdr} % Custom headers and footers
\pagestyle{fancyplain} % Makes all pages in the document conform to the custom headers and footers
\fancyhead{} % No page header - if you want one, create it in the same way as the footers below
\fancyfoot[L]{} % Empty left footer
\fancyfoot[C]{} % Empty center footer
\fancyfoot[R]{\thepage} % Page numbering for right footer
\renewcommand{\headrulewidth}{0pt} % Remove header underlines
\renewcommand{\footrulewidth}{0pt} % Remove footer underlines
\setlength{\headheight}{13.6pt} % Customize the height of the header

\numberwithin{equation}{section} % Number equations within sections (i.e. 1.1, 1.2, 2.1, 2.2 instead of 1, 2, 3, 4)
\numberwithin{figure}{section} % Number figures within sections (i.e. 1.1, 1.2, 2.1, 2.2 instead of 1, 2, 3, 4)
\numberwithin{table}{section} % Number tables within sections (i.e. 1.1, 1.2, 2.1, 2.2 instead of 1, 2, 3, 4)

\setlength\parindent{0pt} % Removes all indentation from paragraphs - comment this line for an assignment with lots of text

%----------------------------------------------------------------------------------------
%	TITLE SECTION
%----------------------------------------------------------------------------------------

\newcommand{\horrule}[1]{\rule{\linewidth}{#1}} % Create horizontal rule command with 1 argument of height

\title{	
\normalfont \normalsize 
\textsc{How to Learn to Code} \\ [25pt] % Your university, school and/or department name(s)
\horrule{0.5pt} \\[0.4cm] % Thin top horizontal rule
\huge R Syllabus \\ % The assignment title
\horrule{2pt} \\[0.5cm] % Thick bottom horizontal rule
}

\author{Amy Pomeroy} % ADD YOUR NAME HERE IF YOU CONTRIBUTE!! 

\date{\normalsize\today} % Today's date or a custom date

\begin{document}

\maketitle % Print the title

%----------------------------------------------------------------------------------------
%	FIRST CLASS
%----------------------------------------------------------------------------------------

\section{First Class - The Basics}

The goal of this class is to introduce the basics of R and get students comfortable working in RStudio. It also serves as a good time to make sure that all students have R and RStudio up and running on their computers. 

%------------------------------------------------

\subsection{Class expectations}

\begin{enumerate}
\item Use the basic math operators (+, -, *, /)
\item Use the assignment operator and how to use it (<-)
\item Understand what a function is, how to use a function, and understand some basic functions
\item Understand the three most common data classes (character, numeric, logical) 
\item Apply the basic comparison operators (>, <, ==, >=, <=)
\item Compare objects, and predict the data classes and how they change when comparing objects 
\end{enumerate}

%----------------------------------------------------------------------------------------
%	SECOND CLASS
%----------------------------------------------------------------------------------------

\section{Second Class - Data Structures}

Be sure to review the information from the previous class (5-10 minutes). Then go over the four basic data structures. Be sure to emphasize the similarities and differences between the data structures. Finally, discuss how to subset each structure, again emphasizing similarities and differences. 

%------------------------------------------------

\subsection{Class expectations}

\begin{enumerate}
\item Understand the basic R data structures (vector, matrix, list, data frame)
\item Subset the four basic data structures 
\end{enumerate}

 %----------------------------------------------------------------------------------------
%	THIRD CLASS
%----------------------------------------------------------------------------------------

\section{Third Class - Plotting Data}

Start this class by introducing how to import data from a csv file. Then review of subsetting by using examples from the imported data, as understanding how to subset the data will make plotting much easier. Then go over the arguments of the basic plot function.  

It would be good if you made a lesson plan for this yourself with data that you find interesting. Please write it up in the same format as the other documents and save it to the GitHub so others can use it. 

%------------------------------------------------

\subsection{Class expectations}

\begin{enumerate}
\item Import data from a csv file format
\item Use the arguments of the plot function
\item Make basic plots  
\end{enumerate}

%----------------------------------------------------------------------------------------
%	FOURTH CLASS
%----------------------------------------------------------------------------------------

\section{Fourth Class - Control Statments}

This is typically the most challenging class for a lot of students. This class does not require a review of plotting to be successful. Make sure to start with very simple examples and only build complexity as the students are understanding. This is a really important concept and takes some patience to teach well.  

%------------------------------------------------

\subsection{Class expectations}

\begin{enumerate}
\item Implement the three basic control statements in R (for-loops, if/else statements, and while statements)
\item Learn the and/or operators for combining logical statements
\end{enumerate}

%----------------------------------------------------------------------------------------
%	FIFTH CLASS
%----------------------------------------------------------------------------------------

\section{Fifth Class - Functions}

If your students are struggling with control loops it would be good to do more control loop practice today and push this lesson back a day. Todays goal is to teach how to write and use functions in R. Be sure to emphasize why they would want to know how to write functions and how functions would be able to help in their research. 

%------------------------------------------------

\subsection{Class expectations}

\begin{enumerate}
\item Write and run a basic function in R
\item Understand function environments and how functions find things
\item Understand the "do not repeat yourself" (DRY) principle 
\end{enumerate}

%----------------------------------------------------------------------------------------
%	SIXTH CLASS
%----------------------------------------------------------------------------------------

\section{Sixth Class - Packages}

You may not reach this lesson if your students struggled with control loops and that's okay. You can alway hand out the lecture notes to those students that are interested. The focus of this lecture is on doing reproducible coding (something we can all work on). 

%------------------------------------------------

\subsection{Class expectations}

\begin{enumerate}
\item Install and load R packages
\item Consider some principles of reproducible research
\item Know the basic components of an R package
\item Create a simple R package using RStudio and roxygen2 
\end{enumerate}

%----------------------------------------------------------------------------------------
%	SEVENTH AND EIGHTH CLASSES
%----------------------------------------------------------------------------------------

\section{Seventh and Eighth Classes - Final Projects}

Devote the last two classes to working on a final project of your choosing. This can be done individually or in groups. Some sample projects will be provided.  

%----------------------------------------------------------------------------------------

\end{document}